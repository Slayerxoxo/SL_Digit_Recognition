%%%%%%%%%%%%%%%%%%%%%%%%%%%%%%%%%%%%%%%%%%%%%%%%%%%%%%%%%%%%%%%%%%%%
%%%%%                         MACROS                           %%%%%
%%%%%%%%%%%%%%%%%%%%%%%%%%%%%%%%%%%%%%%%%%%%%%%%%%%%%%%%%%%%%%%%%%%%
%%%%%                Author : Coraline Marie                   %%%%%
%%%%%%%%%%%%%%%%%%%%%%%%%%%%%%%%%%%%%%%%%%%%%%%%%%%%%%%%%%%%%%%%%%%%

\MakePerPage{footnote} %the perpage package command
%%%%% Ligne de séparation %%%%%
\newcommand{\HRule}{\rule{\linewidth}{0.5mm}}


%%%%% Recoloration des liens %%%%%
\hypersetup{
  backref=true,
  %permet d'ajouter des liens dans...
  pagebackref=true,%...les bibliographies
  hyperindex=true, %ajoute des liens dans les index.
  colorlinks=true, %colorise les liens
  breaklinks=true, %permet le retour à la ligne dans les liens trop longs
  urlcolor=blue, %couleur des hyperliens
  linkcolor=BrickRed, %Couleurs des liens internes
  citecolor=Cyan,
  bookmarks=true, %créé des signets pour Acrobat
  bookmarksopen=true,
  %si les signets Acrobat sont créés,
  %les afficher complètement.
  pdftitle={Signal et langue -- Digit Recognizer}, %informations apparaissant dans
  pdfauthor={Romain RINCE},
  %dans les informations du document
  %sous Acrobat.
}


%%%%% Création de nouvelles couleurs %%%%%
\definecolor{lightgray}{rgb}{0.75,0.75,0.75}


%%%%% Saut de ligne %%%%%
\newcommand{\NewLine}{\vspace{0.5cm}}


%%%%% Inclusion d'en-tête %%%%%
\pagestyle{fancy}
\setlength{\headheight}{23pt}	% Correction du warning \headheight is too small
\lhead{\leftmark}
\rhead{Digit Recognizer}

%%%%% Affichage des numéros de ligne pour la commande lstlisting %%%%%
\lstset{ %
	language=R,
	basicstyle=\footnotesize,
	numbers=left,
	numberstyle=\footnotesize,
	stepnumber=1,
	numbersep=5pt,
	backgroundcolor=\color{white},
	showspaces=false,
	showstringspaces=false,
	showtabs=false,
	frame=single,
	tabsize=2,
	captionpos=b,
	breaklines=true,
	breakatwhitespace=false,
	escapeinside={\%*}{*)}
}

%%%%% Personnalisation du graphe %%%%%
\tikzstyle{state}=[shape=circle,draw=black!50,fill=black!20]
\tikzstyle{observation}=[shape=circle,draw=blue!50,fill=blue!20]
\tikzstyle{lightedge}=[<-,dotted]
\tikzstyle{mainstate}=[state,thick]
\tikzstyle{mainedge}=[<-,thick]
